\documentclass{article}
\usepackage{graphicx} % Required for inserting images

\title{Rozwiązanie problemu pakowania okręgów w kwadrat przy pomocy uczenia maszynowego}
\author{Wojciech Marchewka, Michał Dyrek, Piotr Gwioździk}
\date{\today}


\usepackage{booktabs} % dla lepszej jakości tabel
\usepackage[polish]{babel}
\usepackage[utf8]{inputenc}
\usepackage[T1]{fontenc}
\usepackage{calc}

% Pagination stuff.
\usepackage[letterpaper,top=2cm,bottom=2cm,left=3cm,right=3cm,marginparwidth=1.75cm]{geometry}

% Packages
\usepackage{amsmath}
% \usepackage{graphicx}
\usepackage{float}
\usepackage{amsfonts}
\usepackage[colorlinks=true, allcolors=blue]{hyperref}
\usepackage{titlesec}
\titlelabel{\thetitle.\quad}
\usepackage{xcolor}

\usepackage{amsmath}
\usepackage{listings}
\usepackage{xcolor}
\usepackage{gensymb}
\usepackage{tikz}
\newtheorem{Df}{Definicja}
\newtheorem{Tw}{Twierdzenie}



\definecolor{codegreen}{rgb}{0,0.6,0}
\definecolor{codegray}{rgb}{0.5,0.5,0.5}
\definecolor{codepurple}{rgb}{0.58,0,0.82}
\definecolor{backcolour}{rgb}{0.95,0.95,0.92}

\lstdefinestyle{mystyle}{
    backgroundcolor=\color{backcolour},
    commentstyle=\color{codegreen},
    keywordstyle=\color{magenta},
    numberstyle=\tiny\color{codegray},
    stringstyle=\color{codepurple},
    basicstyle=\normalsize\ttfamily, % Tutaj zmieniamy rozmiar czcionki
    breakatwhitespace=false,
    breaklines=true,
    captionpos=b,
    keepspaces=true,
    numbers=left,
    numbersep=5pt,
    showspaces=false,
    showstringspaces=false,
    showtabs=false,
    tabsize=2
}


\lstset{style=mystyle}

\begin{document}

\maketitle

\section{Wstęp}

W ostatnim czasie wiele najbardziej optymalnych wyników w dziedzinie problemów pakowania (ang. \textit{packing problems}) jest odkrywanych dzięki zastosowaniu metod uczenia maszynowego. 16 maja 2025 roku pracownicy Google opublikowali nowy sposób wykorzystania dużych modeli językowych (LLM) i technik uczenia maszynowego w celu znajdowania rozwiązań problemów pakowania.

Zainspirowani tymi odkryciami w dziedzinie matematyki stosowanej, postanowiliśmy zmierzyć się z problemem pakowania okręgów w kwadracie (\textit{circle packing in a square}). Celem jest rozmieszczenie zadanej liczby jednakowych okręgów w kwadracie o stałym boku w taki sposób, aby ich promień był jak największy, co przekłada się na maksymalizację gęstości pakowania — rozumianej jako stosunek pola wszystkich okręgów do pola kwadratu.

\section{Rozwiązanie problemu różnymi metodami}

Problem postanowiliśmy rozwiązać kilkoma podejściami. Na początku zastosowaliśmy klasyczny algorytm optymalizacyjny SLSQP (ang. \textit{Sequential Least Squares Programming}). Następnie przeszliśmy do metod inspirowanych uczeniem maszynowym, takich jak optymalizacja rojem cząstek (PSO — \textit{Particle Swarm Optimization}) oraz algorytm genetyczny.

Ze względu na to, że czas wykonywania obliczeń rośnie znacząco wraz ze wzrostem liczby okręgów, zdecydowaliśmy się ograniczyć analizę do przypadków od 4 do 10 okręgów.

\subsection{Algorytm optymalizujący — SLSQP}

SLSQP (ang. \textit{Sequential Least Squares Programming}) to klasyczny algorytm optymalizacji numerycznej, przeznaczony do rozwiązywania problemów optymalizacji nieliniowej z ograniczeniami. Znany już od lat 80., znajduje zastosowanie m.in. w zagadnieniach geometrycznych, takich jak problemy pakowania.

\begin{table}[ht]
\centering
\resizebox{\textwidth}{!}{%
\begin{tabular}{rrrrrrr}
\toprule
Liczba okręgów & Promień (SLSQP) & Stosunek (SLSQP) & Gęstość (SLSQP) & Promień (Best) & Stosunek (Best) & Gęstość (Best) \\
\midrule
4  & 0.250000000 & 4.000003000 & 0.785397000 & 0.250000000 & 4.000000000 & 0.785398163 \\
5  & 0.207107000 & 4.828428000 & 0.673765000 & 0.207106700 & 4.828427124 & 0.673765100 \\
6  & 0.187681000 & 5.328201000 & 0.663957000 & 0.187680601 & 5.328201177 & 0.663956909 \\
7  & 0.174458000 & 5.732051000 & 0.669311000 & 0.174457630 & 5.732050807 & 0.669310827 \\
8  & 0.169305000 & 5.906508000 & 0.720407000 & 0.170540600 & 5.863703305 & 0.730963825 \\
9  & 0.166667000 & 6.000000000 & 0.785398000 & 0.166666660 & 6.000000000 & 0.785398163 \\
10 & 0.148182000 & 6.748454000 & 0.689829000 & 0.148204300 & 6.747441523 & 0.690035785 \\
\bottomrule
\end{tabular}%
}
\caption{Porównanie wyników gęstości dla metody SLSQP oraz najlepszych znanych wartości}
\end{table}

\begin{figure}[H]
    \centering
    \includegraphics[width=0.85\linewidth]{SLSQP_1.png}
    \caption{Błąd względny gęstości pakowania uzyskanej metodą SLSQP względem najlepszych znanych wyników}
    \label{fig:slsqp-error}
\end{figure}

Jak pokazuje Tabela~1 oraz wykres na Rysunku 1, algorytm SLSQP poradził sobie bardzo dobrze. Dla prostszych przypadków — takich jak 4, 5, 6 oraz 9 okręgów — uzyskano niemal idealne wyniki. Dla pozostałych przypadków (poza pakowaniem ośmiu okręgów) błąd względny nie przekroczył jednego promila.

\begin{figure}[H]
    \centering
    \begin{minipage}{0.59\linewidth}
        \includegraphics[width=\linewidth]{SLSQP_2.png}
        \caption{Rozmieszczenie 7 okręgów w kwadracie uzyskane metodą SLSQP}
        \label{fig:slsqp-layout}
    \end{minipage}
    \hfill
    \begin{minipage}{0.4\linewidth}
        \includegraphics[width=\linewidth]{SLSQP_3.png}
        \caption{Najlepsze znane rozmieszczenie 7 okręgów w kwadracie}
        \label{fig:best-layout}
    \end{minipage}
\end{figure}

Na Rysunkach 2 oraz 3 przedstawiono porównanie rozmieszczenia siedmiu okręgów uzyskanego za pomocą algorytmu SLSQP z najlepszym znanym rozwiązaniem. Jak widać, nasze podejście bardzo dobrze odwzorowuje optymalną strukturę, co potwierdza skuteczność algorytmu optymalizacyjnego.





\subsection{Algorytm roju cząstek (PSO)}

Pierwszym algorytmem inspirowanym metodami uczenia maszynowego, który zastosowaliśmy do znajdowania optymalnego pakowania okręgów w kwadracie, był algorytm roju cząstek (ang. \textit{Particle Swarm Optimization}, PSO). Wybraliśmy go ze względu na jego skuteczność w zadaniach optymalizacyjnych oraz możliwość uzyskania dobrych wyników poprzez dostrojenie kilku parametrów sterujących.

\begin{table}[ht]
\centering
\resizebox{\textwidth}{!}{%
\begin{tabular}{rrrrrrr}
\toprule
Liczba okręgów & Promień (PSO) & Stosunek (PSO) & Gęstość (PSO) & Promień (Best) & Stosunek (Best) & Gęstość (Best) \\
\midrule
4  & 0.250000000 & 4.000000000 & 0.785398000 & 0.250000000 & 4.000000000 & 0.785398163 \\
5  & 0.207107000 & 4.828427000 & 0.673765000 & 0.207106700 & 4.828427124 & 0.673765100 \\
6  & 0.187678000 & 5.328272000 & 0.663939000 & 0.187680601 & 5.328201177 & 0.663956909 \\
7  & 0.172454000 & 5.798652000 & 0.654024000 & 0.174457630 & 5.732050807 & 0.669310827 \\
8  & 0.168793000 & 5.924429000 & 0.716056000 & 0.170540600 & 5.863703305 & 0.730963825 \\
9  & 0.166667000 & 6.000000000 & 0.785398000 & 0.166666660 & 6.000000000 & 0.785398163 \\
10 & 0.146581000 & 6.822151000 & 0.675005000 & 0.148204300 & 6.747441523 & 0.690035785 \\
\bottomrule
\end{tabular}%
}
\caption{Porównanie wyników gęstości dla PSO i najlepszych znanych wartości}
\end{table}

\begin{figure}[H]
    \centering
    \includegraphics[width=0.85\linewidth]{PSO_1.png}
    \caption{Błąd względny gęstości pakowania uzyskanej metodą PSO względem najlepszych znanych wyników}
    \label{fig:pso-error}
\end{figure}

Jak pokazuje Rysunek 4 oraz Tabela~2, algorytm roju cząstek osiągnął bardzo dobre rezultaty. Podobnie jak metoda SLSQP, PSO uzyskał niemal optymalne rozwiązania dla przypadków z 4, 5, 6 i 9 okręgami. Dla pozostałych przypadków błąd względny nie przekroczył 2\%, co świadczy o wysokiej skuteczności tej metaheurystyki.

\begin{figure}[H]
    \centering
    \begin{minipage}{0.59\linewidth}
        \includegraphics[width=\linewidth]{PSO_2.jpg}
        \caption{Rozmieszczenie 8 okręgów w kwadracie uzyskane metodą PSO}
        \label{fig:pso-layout}
    \end{minipage}
    \hfill
    \begin{minipage}{0.4\linewidth}
        \includegraphics[width=\linewidth]{PSO_3.png}
        \caption{Najlepsze znane rozmieszczenie 8 okręgów w kwadracie}
        \label{fig:best-pso-layout}
    \end{minipage}
\end{figure}

Porównując wyniki uzyskane przez PSO z najlepszymi znanymi rozmieszczeniami (Rysunki 5 i 6), zauważamy pewne różnice, jednak nie są one znaczące. Pokazuje to, że algorytm roju cząstek zdołał znaleźć rozwiązanie bliskie optymalnemu, co potwierdza jego efektywność w problemach pakowania.



\subsection{Algorytm genetyczny}
Drugim algorytmem inspirowanym metodami uczenia maszynowego, który zastosowaliśmy był algorytm genetyczny (ang. 
\section{Porównanie wyników i ich interpretacja}


\section{Bibliografia}
\renewcommand{\refname}{}
\begin{thebibliography}{9}
\bibitem{hydra}
Hydra Project, \emph{Circle Packing in a Square}. Dostęp online: \url{http://hydra.nat.uni-magdeburg.de/packing/csq/csq.html}


\bibitem{shin_kita}
Shin, Y.-B., \& Kita, E. (2012). Solving two-dimensional packing problem using particle swarm optimization. \textit{Computer Assisted Methods in Engineering and Science}, 19(3), 241–255. Dostępne online: \url{https://cames.ippt.pan.pl/index.php/cames/article/download/92/85/}

\bibitem{alphaevolve}
Novikov, A., Vu, N., Eisenberger, M., Dupont, E., Huang, P.-S., Wagner, A. Z., \textit{et al.} (2025). AlphaEvolve: A coding agent for scientific and algorithmic discovery. Google DeepMind. Dostępne online: \url{https://storage.googleapis.com/deepmind-media/DeepMind.com/Blog/alphaevolve-a-gemini-powered-coding-agent-for-designing-advanced-algorithms/AlphaEvolve.pdf}


\bibitem{wiki-csq}
Wikipedia, \emph{Circle packing in a square}. \url{https://en.wikipedia.org/wiki/Circle_packing_in_a_square} 

\bibitem{wiki-packing}
Wikipedia, \emph{Packing problems}. \url{https://en.wikipedia.org/wiki/Packing_problems} 

\bibitem{okulewicz}
M. Okulewicz, \emph{2D Packing and Stock Cutting Problems}. Politechnika Warszawska, dostęp online: \url{https://pages.mini.pw.edu.pl/~okulewiczm/downloads/badania/2dPackingAndStockCutting.pdf} 



\end{thebibliography}


\end{document}
